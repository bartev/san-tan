\documentclass{article}

\usepackage{enumitem}

% achieve very compact lists analogous to the compact paralist environment
\setlist{nolistsep}
% modifies properties of bulleted lists. Here we chose an em dash
% as the label to get a leading wide dash
\setitemize[1]{label=---}
% sets properties valid for numbered lists. We used it to set a
% label and a font for the label
\setenumerate[1]{label=\textcircled{\scriptsize\Alph*},font=\sffamily}
\setenumerate[2]{label={\scriptsize\roman*.},font=\sffamily,start=1}

\begin{document}
\noindent
Look at the subsection in the middle of the list.
The counting continues from where it left off by using

\verb![resume*]!

\vspace{0.5in}

\begin{enumerate}
  \item State the paper size by an option to the document class
  \item Determine the margin dimensions using one of these packages:
  \begin{enumerate}
    \item geometry
    \item typearea
  \end{enumerate}
  \item Customize header and footer by one of these packages:
  \begin{itemize}
    \item fancyhdr
    \item scrpage2
  \end{itemize}
  \end{enumerate}
  \subsubsection*{Tweaking the line spacing:} % (fold)
  \begin{enumerate}[resume*]
  \item Adjust the line spacing for the whole document
  \begin{itemize}
    \item by using the setspace package
    \item or by the command \verb|\linespread{factor}|
  \end{itemize}
\end{enumerate}
\end{document}