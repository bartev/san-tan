\documentclass{article}

\usepackage{amsmath}

\newtheorem{thm}{Theo:}
\newtheorem{dfn}[thm]{Def:}

\begin{document}
\begin{dfn}

A quadratic equation is an equation of the form
\begin{equation}
  \label{quad}
  ax^2 + bx + c = 0,
\end{equation}
where \(a, b\) and \(c \) are constants and $a \neq 0$,
\end{dfn}

\begin{thm}
A quadratic equation (\ref{quad}) has two solutions for the variable $x$:
\begin{equation}
  \label{root}
  x_{1, 2} = \frac{-b \pm \sqrt{b^2 - 4ac}}{2a}.
\end{equation}
\end{thm}
If the \emph{discriminant} $\Delta$ with
$$
\Delta = b^2 - 4 a c
$$
is zero, then the equation (\ref{quad}) has a double solution:
(\ref{root}) becomes
$$
x = -\frac{b}{2a}.
$$

\begin{equation}
   x_1^2 + x_2^2 = 1, \quad 2^{2^x} = 64
\end{equation}

\begin{equation}
   \sqrt[64]{x} = \sqrt{\sqrt{\sqrt{\sqrt{\sqrt{\sqrt{x}}}}}}
\end{equation}
\end{document}